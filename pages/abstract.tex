\begin{abstract}

The emergence of Big Data in various fields in recent years has led to a 
growing need in data processing and an increasing number of 
data-intensive applications.
These massive amount of data and applications must be executed on large-scale 
infrastructures such as cloud or High Performance Computing (HPC) clusters.
However, some relevant challenges remain in resource management , performance, 
scheduling, scalability, etc. 
As a results, there is an increasing demand for performance optimization solutions, 
including data processing. 
While infrastructures with sufficient compute power and storage capacity are available, 
the I/O performance on disks remains as a bottleneck. 
Apart from hardware improvements, the Linux page cache is an efficient approach 
to reduce I/O overheads, but few experimental studies of its interactions with Big Data applications exist, 
partly due to limitations of real-world experiments. 
Simulation is a popular approach to address these issues, however, existing simulation frameworks do not simulate page caching fully, or even at all.  
As a result, simulation-based performance studies of data-intensive applications 
lead to inaccurate results.

This thesis proposes an I/O simulation model that includes
the key features of the Linux page cache. We have implemented this model
as part of the \wrench workflow simulation framework, which itself
builds on the popular \simgrid distributed systems simulation
framework. Our model and its implementation enable the simulation
of both single-threaded and multithreaded applications, and of both
writeback and writethrough caches for local or network-based
filesystems. We evaluate the accuracy of our model in different
conditions, including sequential and concurrent applications, as
well as local and remote I/Os. We find that our page cache model
reduces the simulation error by up to an order of magnitude when
compared to state-of-the-art, cacheless simulations.
\end{abstract}