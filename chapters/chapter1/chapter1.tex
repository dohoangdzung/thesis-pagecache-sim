\chapter{Introduction}
\label{introduction}

\section{Data-intensive applications and HPC}

In the last decades, information technology has been significantly changing the 
world we are living in. 
Thanks to both software and hardware advancements, the application development 
and operational costs have become more reasonable. 
As a result, \english{applications has been widely used} in various fields by companies, 
organizations and individuals. Simultaneously, \english{they} have been collecting a massive 
amount of data at \english{increasing generating speed}.
In addition, the last decade witnessed the rise of Big Data and IoT with the rapid 
development in mobile devices, wearable devices, smart appliances and connectivity  
technologies. Data has increasingly been collected from these sources and transferred 
through network \english{day by day}. 
The \english{ever-increase} in the number and scale of open-data and data sharing initiatives 
have resulted in the evolution in Big Data and data-intensive applications, 
which have been increasingly playing an important role in many fields such as 
science, medical, military, finance, commerce, etc.

Big Data is usually defined with ``three V'' corresponding to \textit{volume}, 
\textit{velocity} and \textit{variety} ~\cite{de2016formal}.  
\english{For \textit{volume}}, the vast amounts of data require scalable and distributed 
approaches for storage solutions as well as querying of information and insights.  
\textit{Velocity} emphasizes on the speed \english{that} data is generated from large data 
streams. Real-time analytics of streaming data requires immediate responses 
from the data ingested by systems.
\textit{Variety} refers to the diversity of data types including both structured and 
unstructured data such as text, audio, images, videos from various data sources.
This data diversity requires not only elastic storage, transfer, processing but also 
compute power to obtain deeper and valuable insights.  
Due to the above challenges in handling the sheer size of the data, data-intensive 
applications must be executed on large-scale infrastructures such as High Performance 
Computing (HPC) clusters or the cloud, which provides a large scale solution 
and parallelism at micro as well as macro level to boost data processing.

Data-intensive applications consist of complex long tasks and workflows 
with a significant amount of data. 
\english{A modification can help to} reduce the execution time by hours of even days.
It is thus crucial to quantify and optimize the performance of these applications 
on HPC platforms. 
However, these systems have heterogeneous and complex architectures since they 
are built by and for different organizations, used for specific purposes, thus must be 
optimized in different ways. 
The goals of optimization of these platforms include determining which type of 
hardware/software stacks are best suited to different application classes, 
as well as understanding the limitations of current algorithms, 
designs and technologies. 
There have been different approaches in order to measure the performance 
of platforms in particular metrics such as execution time, 
resources consumption, power consumption, etc. 

\section{Performance quantification}

Quantifying performance of the platforms basically relies on empirical methods, 
which require running experiments on those platforms. 
There are two common approaches to conduct \english{experiments platforms}, which are 
running real experiments on real platforms and using simulation tools. 
In the first approach, real-world tasks, applications, workflows or pipelines are executed 
directly on a real platform and the information about the performance of the system 
is recorded for later analysis. 
This method can guarantee that the results are realistic since it uses the records 
obtained from real systems with corresponding configurations.
Unfortunately, performance studies relying on real-world experiments on 
compute platforms face many difficulties.
The first challenge is that platforms are normally shared with dynamics loads,
the randomness in the status of the systems that hinder the reproducibility 
of the experiment results.
The next difficulty \english{can be} the labor-intensive experimental setups and 
time-consuming applications, since experiments must be re-executed 
multiple times with multiple settings. 
Moreover, because real-platforms are not built for experimental purposes,
users may not have the freedom to choose platform architectures, software and 
system configurations since some settings may not be permitted and the 
operational cost of the platforms are usually high. 
These shortcomings \english{apparently} preclude the exploration of hypothetical scenarios.

Simulations address these concerns by providing simulation models and 
abstractions for the performance of computer hardware, such as CPU, 
network and storage. 
Simulation models describe the interactions between simulated applications 
and simulated platforms by \english{evolving} application activities (such as computation, 
data transfer) which consume simulated resources (CPU, network, storage) 
throughout simulated application run time~\cite{casanova2014simgrid}. 
As a result, simulations provide a cost-effective, fast, easy and reproducible 
way to evaluate application performance on arbitrary platform configurations. 
It thus comes as no surprise that a large number of simulation frameworks 
have been developed and used for research and development
~\cite{ optorsim, gridsim, groudsim, cloudsim, nunez2012simcan,
nunez2012icancloud, mdcsim, dissect_cf, cloudnetsimplusplus, 
fognetsimplusplus, casanova2014simgrid, ROSS, casanova2020fgcs}. 

\section{Page cache and simulation}

Due to hardware limitations, the performance of platforms can be hampered 
by bottlenecks including data transfer through network links, accessing data 
on disks, etc. 
When it comes to disk I/O, the bottleneck remains as a result of limited 
disk read and write bandwidths, seek time involved in disk accesses, 
and shared bandwidth between processes. 
In data-intensive applications, a significant proportion of execution time 
is spent on data read and write.
Thus, the detrimental effects of I/O bottleneck can be more severe on 
this type of tasks. 
Failing to effectively access data can hinder the performance of the 
whole application, and can lead to wasteful idle time of other resources. 
Although efforts to increase the bandwidth of storage devices with 
advancements in technology, such as solid-state drive (SSD), 
the I/O bottleneck still exists.

Caching is an ubiquitous technique to minimize data transfer from 
low-speed storage by using a high-speed storage layer to store data 
temporarily. Caching reuses previously accessed data for future requests, 
\english{so that it} helps reducing data transfer times. 
Following the idea of caching, page cache is an architectural approach 
implemented in Linux that leverages main memory to improve the 
effectiveness in accessing data on disks, reduce filesystem data transfer times.
As main memory is known to have significant faster read and write bandwidths 
compared to those of disks, the idea behind page cache is to make I/O operations 
occur in memory instead of disks.
With the page cache, previously read data can be kept in cache and then re-read 
directly from memory with memory read bandwidth. 
In writing, data can be written to memory with memory write bandwidth 
before being asynchronously flushed to disk, resulting in better I/O performance 
than on slower storage devices. 
In case the amount of cache available is not sufficient for caching new data, 
data previously read or written to page cache can be flushed to disks or 
evicted from cache with data flushing and cache eviction mechanisms. 
These mechanisms are implemented in the Linux kernel and triggered 
\english{base on particular conditions of} total amount of memory, the amount of data 
being written (i.e., dirty data), the amount of memory available for written data, 
configurations and parameters defined in the kernel, etc.
These parameters combined with the above mentioned flushing and eviction 
mechanisms are the key features of the page cache. 
Hence, these factors should be taken into account when determining the impact 
of I/O on application performance, particularly in data-intensive applications.

Aside from hardware resources, the existence of the page cache in systems has 
considerable impacts on the performance of platforms as it mitigates the I/O 
bottleneck.
As such, it is necessary to have a simulation model of the page cache when 
simulating data-intensive applications in order to achieve higher accuracy  
in simulation results.
While existing simulation frameworks of parallel and distributed computing
systems capture many relevant features of hardware as well as software stacks, 
they lack the ability to simulate page cache with enough details to capture 
key features such as dirty data, data flushing mechanisms and cache eviction 
policies~\cite{nunez2012simcan,nunez2012icancloud}. 
Some simulators, such as the one in~\cite{xu2018saving}, do capture such 
features, but are domain-specific. 

\section{Contributions}

This thesis presents a page cache simulation model with important features 
including data model representing cached data, data flushing mechanism and 
cache eviction policy, which are not fully captured in most  other 
simulation frameworks. 
The thesis also presents simulation algorithms of file read and file write\english{, 
which are} in a chunk-by-chunk manner. 
These functions play the role of the read and write functions in the kernel, 
simulate file read and file write by interacting with the simulated page cache.
We also provide the implementation of the page cache model, 
file I/O algorithms in \wrench~\cite{casanova2020fgcs}, 
a workflow simulation framework based on the popular 
\simgrid distributed simulation toolkit~\cite{casanova2014simgrid}. 
In order to validate and evaluate the accuracy and scalability of the page cache 
simulation model, we implement a simulator in pure Python as well as 
in C with \wrench framework and compare it to a simulator with the original 
WRENCH version, which is not integrated the page cache model. 
The simulator is evaluated in different experiments\english{, which are} common 
scenarios in data-intensive applications on HPC.
The experiments range from single-threaded, multi-threaded applications, 
remote storage application using synthetic data, as well as a real neuroimaging 
pipeline with real data.
The experimental results show that our page cache model significantly improves 
the simulation accuracy by reducing the relative simulation errors by up to 9 times, 
though there are some limitations in the current version of \simgrid framework. 
In addition, our simulator not only achieves more accurate simulation results 
but also simulates correctly the behaviors of the page cache, proves the correctness 
of the model and \english{opens the opportunities to be further enhanced.} 

\section{Thesis organization}

This thesis consists of five main parts divided into chapters.
Chapter~\ref{relatedwork} provides the background knowledge of Linux page cache 
with key features and mechanisms. 
It also summarizes the simulation approach \english{in studies in HPC performance} with 
existing simulation frameworks and models. 
The reasons for the choice of simulation tools is this thesis are explained in this 
chapter as well.
Chapter~\ref{method} presents the page cache simulation model with the key features 
in three main sections.
The first section describes simulation of the page cache components including data 
presentation \TG{not sure what this is}, page cache LRU lists and, flushing and eviction mechanisms.
The second section interprets the algorithms simulating file read and write \english{which work}
in a chunk-by-chunk style.
The last section in this chapter briefly presents the implementation plan to integrate 
the page cache model into WRENCH framework.
Chapter~\ref{experiment} proposes four experiment scenarios to validate the model
including a single-threaded application, a concurrent applications scenario, 
an experiment with remote storage, and a real neuroimaging application. 
The experiment results and evaluation are also presented in this chapter. 
Finally, Chapter~\ref{conclusion} concludes this thesis and discusses the future work. 

