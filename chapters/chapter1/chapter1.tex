\chapter{Introduction}
\label{introduction}

\section{Context and motivation}

The Linux page cache plays an important role in reducing filesystem data 
transfer times. With the page cache, previously read data can be re-read 
directly from memory, and written data can be written to memory before 
being asynchronously flushed to disk, resulting in improved I/O performance
on slower storage devices. The performance improvements depend on many 
factors including the total amount of memory, the amount of data being 
written (i.e., dirty data), and the amount of memory available for
written data. All these factors are important when determining the impact 
of I/O on application performance, particularly in data-intensive applications.

% Swapped the order the two paragraphs and merged them into one
The number of data-intensive applications has been steadily rising as a result of
open-data and data sharing initiatives. Due to the sheer size of the data being
processed, these applications must be executed on large-scale infrastructures
such as High Performance Computing (HPC) clusters or the cloud.  
It is thus crucial to quantify the performance of these applications
on these platforms. The goals include determining which type of hardware/software
stacks are best suited to different application classes, as well asunderstanding 
the limitations of current algorithms, designs and technologies. 
Unfortunately, performance studies relying on real-world experiments on 
compute platforms face several difficulties (high operational costs, 
labor-intensive experimental setups, shared platforms with dynamic loads 
that hinder reproducibility of results) and shortcomings
(experiments are limited to the available platform/software configurations, 
which precludesthe exploration of hypothetical scenarios). 
Simulations address these concerns by providing models and abstractions 
for the performance of computer hardware, such as CPU, network and storage. 
As a result, simulations provide a cost-effective, fast, easy and reproducible 
way to evaluate application performance on arbitrary platform configurations. 
It thus comes as no surprise that a large number of simulation frameworks 
have been developed and used for research and development
~\cite{ optorsim, gridsim, groudsim, cloudsim, nunez2012simcan,
nunez2012icancloud, mdcsim, dissect_cf, cloudnetsimplusplus, fognetsimplusplus, casanova2014simgrid, ROSS, casanova2020fgcs}. 

Page caching is an ubiquitous technique for mitigating the I/O bottleneck.
As such, it is necessary to model it
when simulating data-intensive applications.
While existing simulation frameworks of parallel and distributed computing
systems  capture many relevant features of hardware/software stacks, 
they lack the ability to simulate page cache with enough details to capture key features such
as dirty data and cache eviction policies~\cite{nunez2012simcan,nunez2012icancloud}. 
Some simulators, such as the one in~\cite{xu2018saving}, do capture such features,
but are domain-specific. 

%However, there is a trade-off between accuracy and scalability, which has been
%indicated in some reviews \cite{casanova2014simgrid} \cite{byrne2017review}.
%This means that in most cicumstances, users have to, more or less, sacrifice
%accuracy for the scalability and performance when their platforms grow to
%hundreds or thousands nodes.

\section{Contrubutions}

In this work, we present \wrench-cache, a page cache simulation model
implemented in \wrench~\cite{casanova2020fgcs}, a workflow simulation
framework based on the popular \simgrid distributed simulation
toolkit~\cite{casanova2014simgrid}. Our contributions are:

\begin{itemize}
    \item A page cache simulation model that supports 
both single-threaded and multithreaded applications, and both
writeback and writethrough caches for local or network-based
filesystems;
    \item An implementation of this model in \wrench; 
    \item An evaluation of the accuracy and scalability of our model, and of its implementation,
          for multiple applications, execution scenarios, and page cache configurations. 
\end{itemize}

\section{Thesis organization}
This thesis consists of five main parts divided into chapters.
Chapter~\ref{relatedwork} provides the background knowledge of Linux page cache 
with key features and mechanisms. 
It also summarizes the simulation approach in studies in HPC performance with 
existing simulation frameworks and models. 
The reasons for the choice of simulation tools is this thesis are explained in this 
chapter as well.
Chapter~\ref{method} presents the page cache simulation model with the key features 
in three main sections.
The first section describes simulation of the page cache components including data 
presentation, page cache LRU lists and, flushing and eviction mechanisms.
The second section interprets the algorithms simulating file read and write which work 
in a chunk-by-chunk style.
The last section in this chapter briefly presents the implementation plan to integrate 
the page cache model into WRENCH framework.
Chapter~\ref{experiment} proposes four experiment scenarios to validate the model
including a single-threaded application, a concurrent applications scenario, an experiment 
with remote storage, and a real neuroimaging application. 
The experiment results and evaluation are also presented in this chapter. 
Finally, Chapter~\ref{conclusion} concludes this thesis and discusses the future work. 

