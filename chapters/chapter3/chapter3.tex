\chapter{Methods}
\label{method}

\TG{This chapter should have a more informative title, maybe ``Page cache simulation model''.}

\TG{Add a line of intro: ``This chapter describes our page cache simulation model and its implementation in the WRENCH framework.''}

\TG{The following comments from Review 1 should be taken into account (here and in the paper repo):

\begin{itemize}
\item In Algorithm 2 and 3, the function mm.evict has different calling parameters. What is the difference between these of overloaded methods?

\item Most of the functions called in Algorithm 2 and 3 have same functionality
as presented in the manuscript description of these algorithms but either
the function names or the order of parameters did not match in these two
algorithms. Consistency of the manuscript is key of good presentation.

\item 
Also, it would be easy to understand if the authors provide a diagram which
shows the modifications at the WRENCH design or workflow.
\end{itemize}
}

We separate our simulation model in two components, the I/O
Controller and the Memory Manager, which together simulate
file reads and writes (Figure~\ref{fig:interaction}).
To read or write a file chunk, a simulated application sends a
request to the I/O Controller. The I/O Controller interacts as needed with
the Memory Manager to free memory through flushing or eviction,
and to read or write cached data. The Memory Manager
implements these operations, simulates periodical flushing
and eviction, and reads or writes to disk when necessary.
In case the writethrough strategy is used, the I/O Controller directly writes to disk, 
cache is flushed if needed and written data is added to page cache.

\begin{figure}
       \centering
       \includegraphics[width=0.7\columnwidth]{figures/interaction.pdf}
       \captionof{figure}{Overview of the page cache simulator.
       Applications send file read or write requests to the
       I/O Controller that orchestrates flushing, eviction, cache
       and disk accesses with the Memory Manager. Concurrent accesses to storage
       devices (memory and disk) are simulated using existing models.}
       \label{fig:interaction}
\end{figure}

\input{chapters/chapter3/subsec1}
\input{chapters/chapter3/subsec2}
\input{chapters/chapter3/subsec3}

