\section{I/O Controller}

As mentioned previously, our model reads and writes file chunks in a
round-robin fashion. To read a file chunk, simulated applications send
chunk read requests to the I/O Controller which processes them using
Algorithm~\ref{alg:read}. First, we calculate the amount of uncached
data that needs to be read from disk, and the remaining amount is read
from cache (line 7-8). The amount of memory required to read the chunk
is calculated, corresponding to a copy of the chunk in anonymous memory
and a copy of the chunk in cache (line 9).
If there is not enough available memory, the Memory Manager is called
to flush dirty data (line 10). If necessary, flushing is complemented by
eviction (line 11). Note that, when called with negative arguments, functions
\texttt{flush} and \texttt{evict} simply return and do not do anything. 
Then,if the block requires uncached data, the memory manager is called 
to read data from disk and to add this data to cache (line 14).
If cached data needs to be read, the Memory Manager is called to simulate
a cache read  and update the corresponding data blocks accordingly (line 17).
Finally, the memory manager is called to deallocate the amount of anonymous 
memory used by the application (line 19).

\begin{algorithm}[H]
\caption{File chunk read simulation in I/O Controller}
\label{alg:read}
    \small
    \begin{algorithmic}[1]
        \Input
            \Desc{cs}{chunk size}
            \Desc{fn}{file name}
            \Desc{fs}{file size (assumed to fit in memory)}
            \Desc{mm}{MemoryManager object}
            \Desc{sm}{storage simulation model}
           \EndInput
           \State disk\_read = min(cs, fs - mm.cached(fn)) \Comment{To be read from disk}
           \State cache\_read = cs - disk\_read \Comment{To be read from cache}
           \State required\_mem = cs + disk\_read
           \State mm.flush(required\_mem - mm.free\_mem - mm.evictable)
           \State mm.evict(required\_mem - mm.free\_mem)
           \If {disk\_read $>$ 0}  \Comment{Read uncached data}
           \State sm.read(disk\_read)
           \State mm.add\_to\_cache(disk\_read, fn)
           \EndIf
           \If {cache\_read $>$ 0} \Comment{Read cached}
           \State mm.cache\_read(cache\_read)
        \EndIf
        \State mm.use\_anonymous\_mem(cs)
    \end{algorithmic}
\end{algorithm}

Algorithm~\ref{alg:write} describes our simulation of chunk writes in
the I/O Controller.
% Data is written to cache until the amount of dirty data in the cache
% exceeds the dirty ratio times the amount of available memory. Data is then written to disk.
Our algorithm initially checks the  amount of dirty data that
can be written given the dirty ratio (line 5).
If this amount is greater than 0, the Memory Manager is requested to evict
data from cache if necessary (line 7).
After eviction, the amount of data that can be written to
page cache is calculated (line 8), and a cache write is simulated (line 9).
If the dirty threshold is reached and there is still data to write,
the remaining data is written to cache in a loop
where we repeatedly flush and evict from the cache (line 12-18).

\begin{algorithm}[H]
\caption{File chunk write simulation in I/O Controller}
\label{alg:write}
    \small
    \begin{algorithmic}[1]
        \Input
            \Desc{cs}{chunk size}
            \Desc{fn}{file name}
            \Desc{mm}{MemoryManager object}
           \EndInput
        \State remain\_dirty = dirty\_ratio * mm.avail\_mem - mm.dirty
        \If {remain\_dirty $>$ 0} \Comment{Write to memory}
            \State mm.evict(min(cs, remain\_dirty) - mm.free\_mem)
            \State mem\_amt = min(cs, mm.free\_mem)
            \State mm.write\_to\_cache( mem\_amt, fn)
        \EndIf
        \State remaining = cs - mem\_amt
        \While {remaining $>$ 0}  \Comment{Flush to disk, then write to cache}
            \State mm.flush(cs - mem\_amt)
            \State mm.evict(cs - mem\_amt  - mm.free\_mem)
            \State to\_cache = min(remaining, mm.free\_mem)
            \State mm.write\_to\_cache(to\_cache, fn)
            \State remaining = remaining - to\_cache
        \EndWhile

    \end{algorithmic}
\end{algorithm}

The above model describes page cache in writeback
mode. Our model also includes a write function in writethrough mode,
which simply simulates a disk write with the amount of data passed in,
then evicts cache if needed and adds the written data to the cache.