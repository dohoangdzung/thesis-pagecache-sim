\chapter{Related work}
\label{relatedwork}

\section{Page cache}

\subsection{Page reduces I/O cost}

\subsubsection{The Linux page cache}

To offsets the cost of disk I/O, the Linux kernel implements a disk cache, 
called \textit{page cache}, by storing in memory data that requires disk accesses. 
There are two reasons that make the disk caches important to operating systems. 
First, disk accesses are several orders of magnitude slower than memory accesses. 
Second, there is a likelihood data accessed before will be accessed again 
in near future \cite{linuxdev3rd2010}. 
With memory bandwidth, which is much faster than disk bandwidth, combined with 
the situations that data can be accessed in memory instead of on disk, the I/O 
performance can be largely improved. 
In Linux, the page cache is a part of RAM, which includes physical pages referring 
to pages on disk. 
The size of the page cache is dynamic when it can grow when there is enough 
free memory, and can be shrink to release memory if needed. 

When the kernel starts a read operation, it checks if the required data is in memory. 
If yes, called a \textit{cache hit}, data is then read directly from memory, 
with memory bandwidth, instead of from disk. 
If not, called a \textit{cache miss}, data is read from disk and the kernel 
places a new entry representing this data in the page cache for later reads 
\cite{linuxdev3rd2010}. 
Data cached in the the page cache is pages, which means files must not be 
cached entirely, the page cache can keep the whole file, or only part a file. 

\subsubsection{Writeback and writethrough page cache}

When use of page cache is enabled for a given filesystem, all written pages 
are written first to page cache, prior to being written to disk.
Accessing of these written pages may result in cache hits, should the pages 
remain in memory.
Generally speaking, the page cache can implement one in three 
different write strategies.
In the first strategy, which is \textit{no-write}, the page cache simply does not 
involve in write operations. In \textit{no-write}, data is written directly to disk, 
the cache is invalidated and read from disk for any subsequent requests. 
This strategy is rarely implemented since it not only fails to cache data, 
but also costly invalidates the page cache \cite{linuxdev3rd2010}.
The second strategy is \textit{writethrough}, in which the kernel updates both 
disk and memory cache in write operations. The name \textit{writethrough} 
itself suggests that data is written \textit{through} the page cache to disk with 
disk write bandwidth \cite{linuxdev3rd2010}.
This is a simple solution that can keep data in cache, synchronized between 
page cache and disk, but it does not make the write operations benefit from 
fast memory write bandwidth. 
The third strategy is the approach implemented in Linux kernel, 
called \textit{writeback}. 
With writeback cache, the kernel perform write operations by writing data directly 
into the page cache. However, unlike writethrough, the storage is not immediately 
updated. Instead, the pages that have been written to page cache are marked 
as \textit{dirty} data. These dirty pages are periodically written back to disk by 
a flusher process in predefined intervals. 
In addition, if the kernel needs to reclaim some free memory, it can immediately 
trigger data flushing to write back dirty data to disk. 
After being written back to backing store, these pages are no longer dirty and 
can be removed from page cache when the free memory is insufficient.
The writeback strategy is considered to outperform writethrough as well as
direct I/O (page cache bypassed for I/O) as it delays disk writes to perform 
a bulk write at a later time \cite{linuxdev3rd2010}.

\subsubsection{Caching on NFS}

Data caching requires keeping data close to where it is requested. 
In network filesystem (NFS), data caching means not sending requests 
to server over the network. Thus, data is cached on NFS client cache instead 
of a remote disk \cite{eisler2001managing}. 
In addition, some cache schemes are restricted to ensure data integrity and 
consistency depending on the structure of the filesystem.
In reading on NFS, if data is cached on client, data is read from client cache 
as with local filesystem. If required data is not cached on the client, it will be read 
from server. On server side, the kernel also checks for the availability of data in 
server cache to decide whether data is read from cache or disk. 
However, there is no server cache in writing since the data written to NFS server 
cannot be cached and must be written to disk before the write call on the 
NFS client finishes to ensure data integrity. 
If server cache is enable for writing, if the server crashes during a cache write 
could result in a problem since the client could not be aware of if data has 
been written successfully. 
On the other hand, given a scenario where multiple write operations queue up 
on client side, a client failure before data is written could leave the NFS server 
with an old version of file begin written. 
Thus, only writethrough strategy can be implemented for writing on NFS. 

\subsection{Cache eviction}

Cache eviction and flushing strategies are integral to proper page cache functioning.
Whenever space in memory becomes limited, either as a result of application memory
or page cache use, page cache data may be evicted. Only data that
has been persisted to storage (clean pages) can be flagged for eviction and removed from
memory. Written data that has not yet been persisted to disk (dirty data) must first
be copied (flushed) to storage prior to eviction. When sufficient memory is
being occupied, the flushing process is synchronous. However, even when
there is sufficient available memory, written data will be flushed to disk
at a predefined interval through a process known as \textit{periodical flushing}.
Periodical flushing only flushes expired dirty pages, which remain dirty in
page cache longer than an expiration time configured in the kernel.
Different cache eviction algorithms have also been proposed
\cite{owda2014comparison}.

\subsection{Flushing and periodical flushing}

\subsection{Page cache LRU lists}

\section{Simulation vs experimentation}

The Linux kernel uses a two-list strategy to flag pages for eviction.
The two-list strategy is based on a least recently used (LRU) policy
and uses an active and inactive list in its implementation.
If accessed pages are not in the page cache, they are added to the inactive list.
Should pages located on the inactive list be accessed, they will be moved from
the inactive to the active list.
The lists are also kept balanced by moving pages from the active list
to the inactive list when the active list grows too large.
Thus, the active list only contains pages which are accessed more than once
and not evictable, while the inactive list includes pages accessed once only,
or pages that have been accessed more than once but moved from the active list.
Both lists operate using LRU eviction policies, meaning that data that has
not be accessed recently will be moved first.

\section{Simulation frameworks}

\subsection{Models and concerns}

Many simulation frameworks have been developed to enable the
simulation of parallel and distributed
applications~\cite{optorsim, gridsim, groudsim, cloudsim,
nunez2012simcan,nunez2012icancloud, mdcsim, dissect_cf,
cloudnetsimplusplus, fognetsimplusplus, casanova2014simgrid,
ROSS, casanova2020fgcs}. These frameworks implement simulation
models and abstractions to aid the development of simulators
for studying the functional and performance behaviors of
application workloads executed on various hardware/software
infrastructures. 

The two main concerns for simulation are accuracy,
the ability to faithfully reproduce real-world executions, and
scalability, the ability to simulate large/long real-world
executions quickly and with low RAM footprint. The above
frameworks achieve different compromises between the two.  At
one extreme are discrete-event models that capture
``microscopic'' behaviors of hardware/software systems (e.g.,
packet-level network simulation, block-level disk simulation,
cycle-accurate CPU simulation), which favor accuracy over
speed.  At the other extreme are analytical models that capture
``macroscopic'' behaviors via mathematical models.  While these
models lead to fast simulation, they must be developed
carefully if high levels of accuracy are to be
achieved~\cite{velhoTOMACS2013}. 

In this  work, we use the \simgrid and \wrench simulation
frameworks.  The years of research and development invested in
the popular \simgrid simulation framework~\cite{casanova2014simgrid}, have
culminated in a set of state-of-the-art macroscopic simulation
models that yield high accuracy, as demonstrated by
(in)validation studies and comparisons to competing
frameworks~\cite{smpi_validity, velhoTOMACS2013, simutool_09,
nstools_07, lebre2015, pouilloux:hal-01197274,
smpi_tpds2017,  7885814, 8048921, 7384330}.  But one
significant drawback of \simgrid is that its simulation
abstractions are low-level, meaning that implementing a
simulator of complex systems can be
labor-intensive~\cite{kecskemeti_2014}. To remedy this problem,
the \wrench simulation framework~\cite{casanova2020fgcs}
builds on top of \simgrid to provide higher-level simulation
abstractions, so that simulators of complex applications and
systems can be implemented with a few hundred lines.

\subsection{Existing data caching simulation}

Although the Linux page cache has a large impact on I/O
performance, and thus on the execution of data-intensive
applications, its simulation is rarely considered in the above
frameworks.  Most frameworks merely simulate I/O operations
based on storage bandwidths and capacities.  The SIMCAN
framework does models page caching by storing data accessed on
disk in a block cache~\cite{nunez2012simcan}.  Page cache is
also modeled in iCanCloud through a component that manages
memory accesses and cached data~\cite{nunez2012icancloud}.
However, the scalability of the iCanCloud simulator is limited
as it uses microscopic models.  Besides, none
of these simulators provide any writeback cache simulator nor
cache eviction policies through LRU lists.  Although cache
replacement policies are applied in~\cite{xu2018saving} to
simulate in-memory caching, this simulator is specific to
energy consumption of multi-tier heterogeneous networks.

In this study, we implement a page cache simulation model in the
\wrench framework. We targeted \wrench because it is a recent,
actively developed framework that provides convenient simulation
abstractions, because it is extensible, and because it reuses
\simgrid's scalable and accurate models.

\subsection{SimGrid and WRENCH}
